\documentclass{article}
\usepackage{graphicx} % Required for inserting images


\begin{document}

\title{
Computer Systems and Networks \\ 23W \
         \\ ------------ \\ Topic Proposal: \\ Security in Container Environments \\
}
\date{2023-11-04}
\author{\\Group Work}
\maketitle{}
\bigskip

\renewcommand{\baselinestretch}{0.7}\normalsize
\tableofcontents{}
\renewcommand{\baselinestretch}{1.0}\normalsize

\newpage


%--------------------------------------------------------------------


\section{Team Members}
David Unterholzner (Mat. Nr.: 12009492, d.unterholzner@student.tugraz.at)\\
Jakob Hofer (Mat. Nr.: 12030367, jakob.hofer@student.tugraz.at)\\
Jakob Khom (Mat. Nr.: 12025213, jakob.khom@student.tugraz.at)\\
Leo Lach (Mat. Nr: 12014257, leo.lach@student.tugraz.at)\\

\bigskip
\section{Topic Description}
In our presentation, we will talk about Security in Container Environments. 
To give the audience a complete overview of the concept of containerization and 
the Security aspects behind it, we decided on the following topics that we will discuss:

\begin{enumerate}
    \item \textbf{Understanding Containers:}
    What is a container, and how do they differ from traditional virtual machines?

    \item \textbf{Container vs VM: Different Use Cases and Engines:}
    Explore the distinctive use cases for containers and virtual machines. 
    Talk about the most popular Container Engine (Docker).

    \item \textbf{Achieving Isolation in Containers:}
    Explain the mechanisms behind container isolation, including Linux namespaces, 
    capabilities, and control groups (cgroups).

    \item \textbf{Runtime and Network Security:}
    Talk about runtime protection and network security within container environments.

    \item \textbf{Best Practices for Building Container Images:}
    Talk about the best practices for constructing secure container images, 
    ensuring a solid foundation for deployment.

    \item \textbf{Learning from Past Exploits:}
    Analyze real-world security exploits related to containers, 
    providing insights into preventive measures and lessons learned.

    \item \textbf{Programming Example:}
    Illustrate security principles through a practical programming example and showcase how 
    a container is isolated from the rest of the System.
\end{enumerate}

\bigskip
\section{Description and plan for the code example}

\begin{enumerate}
\item \textbf{Filesystem restrictions:} Examples of using binds/volumes to share data between container and host, 
using/escaping from chroot.
\item \textbf{Resource limitation:} Show that programs running in a container can't use up all host memory/cpus.
The code will contain a C program with a memory leak and a python program with ReDOS.
\item \textbf{Syscall limitation:} A C program that runs syscalls that are blocked by default (e.g. reboot, ptrace), 
and an exploit for breaking out of a misconfigured container (unconfined seccomp).
\item \textbf{Container Image Integrity Check:} A C program that creates a hash of a container image and 
compares that hash to a well-known hash of the image to ensure the image hasn't been tampered with.
\end{enumerate}

\end{document}
